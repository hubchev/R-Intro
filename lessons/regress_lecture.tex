% Options for packages loaded elsewhere
\PassOptionsToPackage{unicode}{hyperref}
\PassOptionsToPackage{hyphens}{url}
%
\documentclass[
]{article}
\usepackage{lmodern}
\usepackage{amssymb,amsmath}
\usepackage{ifxetex,ifluatex}
\ifnum 0\ifxetex 1\fi\ifluatex 1\fi=0 % if pdftex
  \usepackage[T1]{fontenc}
  \usepackage[utf8]{inputenc}
  \usepackage{textcomp} % provide euro and other symbols
\else % if luatex or xetex
  \usepackage{unicode-math}
  \defaultfontfeatures{Scale=MatchLowercase}
  \defaultfontfeatures[\rmfamily]{Ligatures=TeX,Scale=1}
  \setmainfont[]{Calibri Light}
\fi
% Use upquote if available, for straight quotes in verbatim environments
\IfFileExists{upquote.sty}{\usepackage{upquote}}{}
\IfFileExists{microtype.sty}{% use microtype if available
  \usepackage[]{microtype}
  \UseMicrotypeSet[protrusion]{basicmath} % disable protrusion for tt fonts
}{}
\makeatletter
\@ifundefined{KOMAClassName}{% if non-KOMA class
  \IfFileExists{parskip.sty}{%
    \usepackage{parskip}
  }{% else
    \setlength{\parindent}{0pt}
    \setlength{\parskip}{6pt plus 2pt minus 1pt}}
}{% if KOMA class
  \KOMAoptions{parskip=half}}
\makeatother
\usepackage{xcolor}
\IfFileExists{xurl.sty}{\usepackage{xurl}}{} % add URL line breaks if available
\IfFileExists{bookmark.sty}{\usepackage{bookmark}}{\usepackage{hyperref}}
\hypersetup{
  pdftitle={Regression Analysis},
  pdfauthor={Stephan.Huber@hs-fresenius.de},
  hidelinks,
  pdfcreator={LaTeX via pandoc}}
\urlstyle{same} % disable monospaced font for URLs
\usepackage[margin=1in]{geometry}
\usepackage{color}
\usepackage{fancyvrb}
\newcommand{\VerbBar}{|}
\newcommand{\VERB}{\Verb[commandchars=\\\{\}]}
\DefineVerbatimEnvironment{Highlighting}{Verbatim}{commandchars=\\\{\}}
% Add ',fontsize=\small' for more characters per line
\usepackage{framed}
\definecolor{shadecolor}{RGB}{248,248,248}
\newenvironment{Shaded}{\begin{snugshade}}{\end{snugshade}}
\newcommand{\AlertTok}[1]{\textcolor[rgb]{0.94,0.16,0.16}{#1}}
\newcommand{\AnnotationTok}[1]{\textcolor[rgb]{0.56,0.35,0.01}{\textbf{\textit{#1}}}}
\newcommand{\AttributeTok}[1]{\textcolor[rgb]{0.77,0.63,0.00}{#1}}
\newcommand{\BaseNTok}[1]{\textcolor[rgb]{0.00,0.00,0.81}{#1}}
\newcommand{\BuiltInTok}[1]{#1}
\newcommand{\CharTok}[1]{\textcolor[rgb]{0.31,0.60,0.02}{#1}}
\newcommand{\CommentTok}[1]{\textcolor[rgb]{0.56,0.35,0.01}{\textit{#1}}}
\newcommand{\CommentVarTok}[1]{\textcolor[rgb]{0.56,0.35,0.01}{\textbf{\textit{#1}}}}
\newcommand{\ConstantTok}[1]{\textcolor[rgb]{0.00,0.00,0.00}{#1}}
\newcommand{\ControlFlowTok}[1]{\textcolor[rgb]{0.13,0.29,0.53}{\textbf{#1}}}
\newcommand{\DataTypeTok}[1]{\textcolor[rgb]{0.13,0.29,0.53}{#1}}
\newcommand{\DecValTok}[1]{\textcolor[rgb]{0.00,0.00,0.81}{#1}}
\newcommand{\DocumentationTok}[1]{\textcolor[rgb]{0.56,0.35,0.01}{\textbf{\textit{#1}}}}
\newcommand{\ErrorTok}[1]{\textcolor[rgb]{0.64,0.00,0.00}{\textbf{#1}}}
\newcommand{\ExtensionTok}[1]{#1}
\newcommand{\FloatTok}[1]{\textcolor[rgb]{0.00,0.00,0.81}{#1}}
\newcommand{\FunctionTok}[1]{\textcolor[rgb]{0.00,0.00,0.00}{#1}}
\newcommand{\ImportTok}[1]{#1}
\newcommand{\InformationTok}[1]{\textcolor[rgb]{0.56,0.35,0.01}{\textbf{\textit{#1}}}}
\newcommand{\KeywordTok}[1]{\textcolor[rgb]{0.13,0.29,0.53}{\textbf{#1}}}
\newcommand{\NormalTok}[1]{#1}
\newcommand{\OperatorTok}[1]{\textcolor[rgb]{0.81,0.36,0.00}{\textbf{#1}}}
\newcommand{\OtherTok}[1]{\textcolor[rgb]{0.56,0.35,0.01}{#1}}
\newcommand{\PreprocessorTok}[1]{\textcolor[rgb]{0.56,0.35,0.01}{\textit{#1}}}
\newcommand{\RegionMarkerTok}[1]{#1}
\newcommand{\SpecialCharTok}[1]{\textcolor[rgb]{0.00,0.00,0.00}{#1}}
\newcommand{\SpecialStringTok}[1]{\textcolor[rgb]{0.31,0.60,0.02}{#1}}
\newcommand{\StringTok}[1]{\textcolor[rgb]{0.31,0.60,0.02}{#1}}
\newcommand{\VariableTok}[1]{\textcolor[rgb]{0.00,0.00,0.00}{#1}}
\newcommand{\VerbatimStringTok}[1]{\textcolor[rgb]{0.31,0.60,0.02}{#1}}
\newcommand{\WarningTok}[1]{\textcolor[rgb]{0.56,0.35,0.01}{\textbf{\textit{#1}}}}
\usepackage{graphicx,grffile}
\makeatletter
\def\maxwidth{\ifdim\Gin@nat@width>\linewidth\linewidth\else\Gin@nat@width\fi}
\def\maxheight{\ifdim\Gin@nat@height>\textheight\textheight\else\Gin@nat@height\fi}
\makeatother
% Scale images if necessary, so that they will not overflow the page
% margins by default, and it is still possible to overwrite the defaults
% using explicit options in \includegraphics[width, height, ...]{}
\setkeys{Gin}{width=\maxwidth,height=\maxheight,keepaspectratio}
% Set default figure placement to htbp
\makeatletter
\def\fps@figure{htbp}
\makeatother
\setlength{\emergencystretch}{3em} % prevent overfull lines
\providecommand{\tightlist}{%
  \setlength{\itemsep}{0pt}\setlength{\parskip}{0pt}}
\setcounter{secnumdepth}{-\maxdimen} % remove section numbering
\usepackage{amsmath}

\title{Regression Analysis}
\author{\href{mailto:Stephan.Huber@hs-fresenius.de}{\nolinkurl{Stephan.Huber@hs-fresenius.de}}}
\date{December 2020 -- HS-Fresenius: Data Science Course}

\begin{document}
\maketitle

{
\setcounter{tocdepth}{4}
\tableofcontents
}
\begin{center}\rule{0.5\linewidth}{0.5pt}\end{center}

\hypertarget{why-to-care-about-regression-analysis}{%
\subsection{Why to care about regression
analysis}\label{why-to-care-about-regression-analysis}}

\includegraphics{~/Dropbox/hsf/pic/destat/afact2.gif}

\begin{itemize}
\tightlist
\item
  Regressions allow us to draw insights from data,
\item
  to analyze and interpret the strength of relationships and
\item
  to reduce the likeliness of causal fallacy.
\end{itemize}

\begin{center}\rule{0.5\linewidth}{0.5pt}\end{center}

\begin{itemize}
\item
  \textbf{Linear regression} is a predictive modeling techniques that
  aims to find a mathematical equation for a variable \(y\) as a
  function of one (simple linear model) or more variables (multiple
  linear regression), \(x\).
\item
  The method to \textit{fit a line} is called the ordinary least squared
  (OLS) method as it minimizes the sum of the squared differences of all
  \(y_i\) and \(y_i^*\) as sketched below.
\end{itemize}

\includegraphics[width=0.5\textwidth,height=\textheight]{~/Dropbox/hsf/pic/destat/regression_ols.png}

\begin{center}\rule{0.5\linewidth}{0.5pt}\end{center}

The simple linear regression model is \[
y_i = \beta_{0} + \beta_{1} x_i + \epsilon_i
\] where

\begin{itemize}
\tightlist
\item
  the index \(i\) runs over the observations, \(i=1,\dots,n\)
\item
  \(y_i\) is the \textbf{dependent variable}, the regressand
\item
  \(x_i\) is the \textbf{independent variable}, the regressor
\item
  \(\beta_0\) is the \textbf{intercept} or constant
\item
  \(\beta_1\) is the slope of regression line
\item
  \(\epsilon_i\) is the \textbf{error term} or the residual.
\end{itemize}

\begin{center}\rule{0.5\linewidth}{0.5pt}\end{center}

\hypertarget{ols-estimation-method}{%
\subsection{OLS estimation method}\label{ols-estimation-method}}

\begin{itemize}
\tightlist
\item
  minimize the squared residuals by choosing the estimated coefficients
  \(\hat{\beta_{0}}\) and \(\hat{\beta_{1}}\)
\end{itemize}

\$ \min\emph{\{\hat{\beta_{0}}, \hat{\beta_{1}}\}\sum}\{i=1\}
\epsilon\emph{i\^{}2 \&= \sum}\{i=1\}
\left[y_i - \underbrace{(\hat{\beta_{0}} + \hat{\beta_{1}} x_i)}_{\textnormal{predicted values}\equiv y_i^*}\right]\^{}2
\$

\[
\Leftrightarrow  &=  \sum_{i=1} (y_i - \hat{\beta_{0}} - \hat{\beta_{1}} x_i)^2
\]

\begin{itemize}
\tightlist
\item
  Minimizing the function requires to calculate the first order
  conditions with respect to alpha and beta and set them zero:
  \begin{align*}
  \frac{\partial \sum_{i=1} \epsilon_i^2}{\partial \beta_{0}}=2 \sum_{i=1}    (y_i - \hat{\beta_{0}} - \hat{\beta_{1}} x_i)=0\\
  \frac{\partial \sum_{i=1} \epsilon_i^2}{\partial \beta_{1}}=2 \sum_{i=1}    (y_i - \hat{\beta_{0}} - \hat{\beta_{1}} x_i)x_i=0
  \end{align*}
\item
  This is just a linear system of two equations with two unknowns
  \(\beta_{0}\) and \(\beta_{1}\), which we can mathematically solve for
  \(\beta_0\): \begin{align*}
   &\sum_{i=1} (y_i - \hat{\beta_{0}} - \hat{\beta_{1}} x_i)=0\\
   \Leftrightarrow \hat{\beta_{0}}&=\frac{1}{n}\sum_{i=1}  (y_i  - \hat{\beta_{1}} x_i)\\
   \Leftrightarrow \hat{\beta_{0}}&=\bar{y}-\hat{\beta_{1}}\bar{x}
  \end{align*}
\item
  and for \(\beta_{1}\): \begin{align*}
   &\sum_{i=1} (y_i - \hat{\beta_{0}} - \hat{\beta_{1}} x_i)x_i=0\\
   \Leftrightarrow & \sum_{i=1}    y_i x_i- \underbrace{\hat{\beta_{0}}}_{\bar{y}-\hat{\beta_{1}}\bar{x}}x_i - \hat{\beta_{1}} x_i^2=0\\
   \Leftrightarrow & \sum_{i=1}    y_i x_i- (\bar{y}-\hat{\beta_{1}}\bar{x})x_i - \hat{\beta_{1}} x_i^2=0\\    
   \Leftrightarrow & \sum_{i=1}    y_i x_i- \bar{y}x_i-\hat{\beta_{1}}\bar{x}x_i - \hat{\beta_{1}} x_i^2=0\\   
   \Leftrightarrow & \sum_{i=1}    (y_i - \bar{y}-\hat{\beta_{1}}\bar{x} - \hat{\beta_{1}} x_i)x_i=0\\ 
  %   \Leftrightarrow & \sum_{i=1}    (y_i - \bar{y})-\beta_{1}\bar{x} - \hat{\beta_{1}} x_i=0\\
   \Leftrightarrow & \sum_{i=1} (y_i - \bar{y}) x_i -\hat{\beta_{1}}(\bar{x} -  x_i)x_i =0\\
   \Leftrightarrow & \sum_{i=1}    (y_i - \bar{y}) x_i  =  \hat{\beta_{1}} \sum_{i=1} (\bar{x} -  x_i) x_i \\
  %   \Leftrightarrow & \beta_{1} =\frac{\sum_{i=1}(y_i - \bar{y})x_i }{ \sum_{i=1} (\bar{x} -  x_i)x_i }\\
   \Leftrightarrow & \hat{\beta_{1}} =\frac{\sum_{i=1}(y_i - \bar{y})x_i }{ \sum_{i=1} (\bar{x} -  x_i)x_i }\\
   \Leftrightarrow & \hat{\beta_{1}} =\frac{\sum_{i=1}(y_i -\bar{y})(x_i-\bar{x})}{\sum_{i=1} (\bar{x} -  x_i)^2 }\\
   \Leftrightarrow & \hat{\beta_{1}} ={\frac {\sigma_{x,y}}{\sigma^2_{x}}}
  \end{align*}
\end{itemize}

\begin{center}\rule{0.5\linewidth}{0.5pt}\end{center}

\begin{Shaded}
\begin{Highlighting}[]
\KeywordTok{summary}\NormalTok{(cars)}
\end{Highlighting}
\end{Shaded}

\begin{verbatim}
##      speed           dist       
##  Min.   : 4.0   Min.   :  2.00  
##  1st Qu.:12.0   1st Qu.: 26.00  
##  Median :15.0   Median : 36.00  
##  Mean   :15.4   Mean   : 42.98  
##  3rd Qu.:19.0   3rd Qu.: 56.00  
##  Max.   :25.0   Max.   :120.00
\end{verbatim}

\hypertarget{slide-with-plot}{%
\subsection{Slide with Plot}\label{slide-with-plot}}

\includegraphics{regress_lecture_files/figure-latex/pressure-1.pdf}

\end{document}
